\documentclass[9pt,twocolumn]{opticajnl}
\journal{opticajournal}
\setboolean{shortarticle}{true}

% Keep typography and packages consistent with the main Optica letter
\usepackage{siunitx}
\sisetup{range-phrase = --, range-units = single}
\usepackage{xcolor}

\title{Supplementary Material: Self-Calibrated Neuromorphic Hyperspectral Sensing}
\author{Rongzhou Chen}
\author{Chutian Wang}
\author{Yuqing Cao}
\author[*]{Edmund Y. Lam}
\affil{Department of Electrical and Electronic Engineering, The University of Hong Kong, Pokfulam, Hong Kong SAR, China}
\affil[*]{elam@eee.hku.hk}
\dates{\today}

\begin{abstract}
This document provides supplementary methods, implementation notes, and additional figures supporting the main manuscript titled ``Self-Calibrated Neuromorphic Hyperspectral Sensing.'' It follows the Optica journal letter template and reuses the same notation and plotting conventions.
\end{abstract}

\begin{document}

\maketitle

% Simple section helper matching main style
\newcommand{\olsection}[1]{\par\noindent\textbf{#1.} }

\olsection{Additional Methods}
We briefly summarize details that complement the main text:
\begin{itemize}
  \item Scan segmentation uses the activity auto-correlation and auto-convolution peaks to determine the one-way period and pre/post regions.
  \item Multi-window compensation fits linear boundary surfaces per window and optimizes spatial variance within fixed temporal bins to undo temporal shear.
  \item Spectral displays use 50~ms cumulative bins unless otherwise noted; background subtraction and light smoothing may be applied for visibility.
\end{itemize}

\olsection{Supplementary Figure S1}
Figure~\ref{fig:s1} shows the aligned background (event-derived) against a spectrometer curve, formatted as in the main text.

\begin{figure}[t]
  \centering
  \includegraphics[width=\columnwidth]{figures/figure04_rescaled_bg_gt_third_only.pdf}
  \caption{Supplementary Figure S1: Background mapped to wavelength (blue) overlaid with Light SPD (orange). Legend placed inside, consistent with Fig.~5 styling.}
  \label{fig:s1}
\end{figure}

\olsection{Reproducibility Notes}
The following commands (run in the \texttt{nhi\_test} environment) regenerate the figures referenced here:
\begin{itemize}
  \item Figure 4 grid and overlay:
\texttt{python publication\_code/figure04\_rescaled\_allinone.py \\
  --segment scan\_angle\_20\_led\_2835b/angle\_20\_sanqin\_2835\_20250925\_184638/angle\_20\_sanqin\_2835\_event\_20250925\_184638\_segments/Scan\_1\_Forward\_events.npz \\
  --gt-dir groundtruth\_spectrum\_2835 \\
  --gt-frames-dir hyperspectral\_data\_sanqin\_gt/test300\_roi\_square\_frames\_matched \\
  --bin-width-us 50000 --start-bin 3 --end-bin 15 --show-wavelength \\
  --add-gt-row --bar-height-ratio 0.06 --bar-px 4 --save-png}
\end{itemize}

\bibliography{ref}

\end{document}

