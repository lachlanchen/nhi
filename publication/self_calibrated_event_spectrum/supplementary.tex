\documentclass[9pt,twocolumn]{opticajnl}
\journal{opticajournal}
\setboolean{shortarticle}{true}
% Load siunitx when available; otherwise provide minimal fallbacks so local builds succeed
\IfFileExists{siunitx.sty}{%
  \usepackage{siunitx}
  \sisetup{range-phrase = --, range-units = single}
  \sisetup{per-mode = symbol}
  \DeclareSIUnit{\pixel}{px}
  \DeclareSIUnit{\USD}{USD}
}{%
  \newcommand{\sisetup}[1]{}
  % When defining macros inside an \IfFileExists argument, use doubled #'s
  \newcommand{\SI}[2]{\ensuremath{##1\,##2}}
  \newcommand{\SIrange}[3]{\ensuremath{##1\text{--}##2\,##3}}
  \newcommand{\DeclareSIUnit}[2]{}
  % Common unit symbols used in this manuscript
  \providecommand{\nano}{\ensuremath{\mathrm{n}}}
  \providecommand{\micro}{\ensuremath{\mu}}
  \providecommand{\milli}{\ensuremath{\mathrm{m}}}
  \providecommand{\second}{\ensuremath{\mathrm{s}}}
  \providecommand{\meter}{\ensuremath{\mathrm{m}}}
  \providecommand{\per}{\ensuremath{\,/\,}}
  \providecommand{\pixel}{\ensuremath{\mathrm{px}}}
  \providecommand{\USD}{\ensuremath{\mathrm{USD}}}
}
\usepackage{ifthen}
\usepackage{placeins}
\usepackage{xcolor}
% Ensure citation commands (\cite) are available in local builds
\IfFileExists{natbib.sty}{\usepackage[numbers,sort&compress]{natbib}}{}
\usepackage{tikz}
\usetikzlibrary{arrows.meta,positioning,shadings,calc}
% Allow graphics to be found in multiple common output folders
\usepackage{graphicx}
\graphicspath{{figures/}{figures_sm/}{../figures/}{../../publication_code/figures/}}

% Robust includegraphics helper: prefer PNG, then PDF; otherwise draw a placeholder box
\newcommand{\includegraphicsSafe}[2]{%% #1 base name or path (without extension), #2 width
  \IfFileExists{#1.png}{\includegraphics[width=#2]{#1.png}}{%%
    \IfFileExists{#1.pdf}{\includegraphics[width=#2]{#1.pdf}}{%%
      \IfFileExists{#1.jpg}{\includegraphics[width=#2]{#1.jpg}}{%%
        \IfFileExists{#1.jpeg}{\includegraphics[width=#2]{#1.jpeg}}{%%
          \fbox{\parbox{#2}{\centering\vspace{1em}\footnotesize Missing figure\vspace{1em}}}%%
        }%%
      }%%
    }%%
  }%%
}

% color is already loaded above; keep typography consistent with main letter

\title{Supplementary Material: Self-Calibrated Neuromorphic Hyperspectral Sensing}
\author[]{Rongzhou Chen}
\author[]{Chutian Wang}
\author[]{Yuqing Cao}
\author[*]{Edmund Y. Lam}
\affil[]{Department of Electrical and Electronic Engineering, The University of Hong Kong, Pokfulam, Hong Kong SAR, China}
\affil[*]{elam@eee.hku.hk}
\dates{\today}

\begin{abstract}
This document provides supplementary methods, implementation notes, and additional figures supporting the main manuscript titled ``Self-Calibrated Neuromorphic Hyperspectral Sensing.'' 
\end{abstract}

\begin{document}

\maketitle

% Simple section helper matching main style
\newcommand{\olsection}[1]{\par\noindent\textbf{#1.} }

\olsection{Event-trigger probability model}
Event emission occurs when the logarithmic brightness increment surpasses a fixed magnitude $\theta$. Under realistic sensing conditions the observed gradient is corrupted by noise, which we model as
\begin{equation}
  \nabla I = \nabla I_{\text{true}} + \eta,\qquad \eta \sim \mathcal{N}(0,\sigma^2).
  \label{eq:gradient_noise}
\end{equation}

The probability of a positive event equals the likelihood that the noise-perturbed gradient exceeds the threshold,
\begin{equation}
  P(\text{event}) = P(\eta > \theta - \nabla I_{\text{true}}) = \tfrac12\!\left[1 - \operatorname{erf}\!\left(\frac{\theta - \nabla I_{\text{true}}}{\sqrt{2}\sigma}\right)\right],
  \label{eq:trigger_probability}
\end{equation}
with an analogous form for negative events around $-\theta$. Approximating the error function by a logistic curve yields the computationally convenient sigmoid expression
\begin{equation}
  P(\text{event}) \approx \sigma\!\left(\frac{\nabla I_{\text{true}} - \theta}{\sigma'}\right),\qquad \sigma' \approx \frac{2\sqrt{2}}{\sqrt{\pi}}\sigma,
  \label{eq:sigmoid_probability}
\end{equation}
where $\sigma(\cdot)$ denotes the logistic sigmoid. Within a temporal bin of duration $\Delta t$ the expected event count relates directly to this probability,
\begin{equation}
  \langle N_{\text{evt}}\rangle = \lambda\,P(\text{event})\,\Delta t,
  \label{eq:expected_events}
\end{equation}
with $\lambda$ the effective sampling rate per pixel. Averaging events over repeated scans gives the empirical mean $\bar{N}_{\text{evt}}$, which can be inverted via the logit function $\sigma^{-1}$ to estimate the true gradient:
\begin{equation}
  \nabla I_{\text{est}} = \theta + \sigma'\,\sigma^{-1}\!\left(\frac{\bar{N}_{\text{evt}}}{\lambda\,\Delta t}\right).
  \label{eq:gradient_estimate}
\end{equation}

This probabilistic link between event statistics and intensity gradients underpins the reconstruction strategy in the main letter. Equation~(\ref{eq:gradient_estimate}) justifies using temporally binned events as sufficient statistics for spectral estimation: the scan-induced evolution of $\nabla I_{\text{true}}$ encodes spectral slopes, while averaging suppresses sensor noise and preserves the dynamic-range benefits of event vision. Empirically, this behaviour is visible in Fig.~\ref{fig:event_probability}: across a \SI{3.5}{\second} scan, 10~ms accumulations exhibit spatial event densities that deviate from the binary extremes and vary with the local signed temporal derivative of brightness (positive derivatives yield predominantly positive events; negative derivatives yield predominantly negative events). Related evidence that event streams encode log-intensity changes has also been observed in static-scene recovery from noise-enabled triggering~\cite{cao2025noise2image}. Subsequent steps segment scans and correct residual temporal shear before building the spectral cube.

\begin{figure*}[t]
  \centering
  \includegraphicsSafe{figures_sm/event_probability_results}{0.98\textwidth}\\[-2pt]
  \caption{Empirical event probabilities under scanning illumination. Shown are example 10~ms accumulations sampled across a single \SI{3.5}{\second} scan. The spatial event density and polarity vary smoothly with the local signed temporal derivative of brightness: regions with increasing intensity emit predominantly positive events (red), decreasing regions emit negative events (blue), and near-stationary regions remain close to zero density. This observation is consistent with the model in Eqs.~(\ref{eq:trigger_probability})--(\ref{eq:expected_events}), in which per-bin event probabilities deviate from 0 and 1 according to the magnitude and sign of $\nabla I_{\text{true}}$, thereby encoding spectral slope information.}
  \label{fig:event_probability}
\end{figure*}

As a complementary, source-referenced validation, Fig.~\ref{fig:cumsum_vs_spd} examines how background event statistics relate to the illumination spectrum during a \SI{3.5}{\second} scan. The top row visualizes properties of the calibrated Lumileds spectral power distribution (SPD): (a) the derivative of the log-SPD induced by the wavelength--time mapping, (b) the log-SPD itself, and (c) the original SPD. The bottom row shows background event measurements formed by counting polarity-weighted events in 5~ms bins across the scan: (d) the normalized dominant background event density, (e) its cumulative sum, and (f) the exponential of the normalized cumulative sum. Consistent with the probabilistic trigger model, the background density in (d) qualitatively follows the derivative of log intensity in (a), while the integrated-and-exponentiated curve in (f) recovers the SPD shape in (c) up to a global scale.

\begin{figure}[t]
  \centering
  \includegraphicsSafe{figures_sm/cumsum_vs_spd}{\columnwidth}\\[-2pt]
  \caption{Relationship between light-source spectrum and background events. Top row: properties of the Lumileds spectral power distribution (SPD)---(a) derivative of the log-SPD, (b) log-SPD, and (c) original SPD. Bottom row: background event statistics over a \SI{3.5}{\second} scan with 5~ms bins---(d) normalized dominant background event density, (e) cumulative sum of the density, and (f) exponential of the normalized cumulative sum. The correspondence between (a) and (d), and between (c) and (f), indicates that the background event stream encodes the source SPD up to integration and an exponential mapping.}
  \label{fig:cumsum_vs_spd}
\end{figure}

% \FloatBarrier

\olsection{Self-synchronized scan segmentation}
We summarize the segmentation procedure used in the main letter and provide implementation details without repeating its narrative description. Event timestamps are first binned into an activity signal $a[n]$ (events per 1~ms bin). The peak of auto-correlation
\begin{equation}
  R[k] = \sum_{n} a[n]\,a[n+k]
\end{equation}
reveals the scanning period, while the auto-convolution of $a[n]$ the correlation of $a[n]$ and its time-reversed sequence $a_{\mathrm{rev}}[n] = a[N{-}1{-}n]$,
\begin{equation}
  R_{\mathrm{rev}}[k] = \sum_{n} a[n]\,a[N{-}1{-}n],
  \label{eq:rev_corr}
\end{equation}
exposes the pre- and post-scan regions through mirror-symmetric peaks around the turnaround. An iterative peak search then refines the start and end indices of each half-scan, producing forward/backward segments without auxiliary triggers. The main letter includes illustrative plots of these signals; here we focus on the core computations to keep the supplement self-contained.

This data-driven segmentation adapts to slow drifts or perturbations and yields consistent temporal support for reconstruction.

\olsection{Multi-window scanning compensation}
Even with accurate segmentation, non-uniform motor motion induces spatially varying temporal delays and shear in the event cloud. Our multi-window compensation routine addresses this by fitting boundary surfaces
\begin{equation}
  T_i(x,y) = a_i x + b_i y + c_i,
\end{equation}
which partition each scan into $M$ temporal windows. Soft memberships,
\begin{equation}
  w_i(x,y,t) = \frac{\sigma\!\left(\frac{t - T_i}{\tau}\right)\sigma\!\left(\frac{T_{i+1}-t}{\tau}\right)}{\sum_j \sigma\!\left(\frac{t - T_j}{\tau}\right)\sigma\!\left(\frac{T_{j+1}-t}{\tau}\right)},
\end{equation}
blend adjacent windows (temperature $\tau \approx 1$~ms). The compensated time is
\begin{equation}
  t' = t - \sum_{i=1}^{M-1} w_i(x,y,t)\bigl(\tilde{a}_i x + \tilde{b}_i y \bigr),
  \label{eq:timewarp}
\end{equation}
where $\{\tilde{a}_i,\tilde{b}_i\}$ are trainable slopes optimized by minimizing the spatial variance of binned event intensities while enforcing smoothness across neighbouring windows. The objective is 
\begin{equation}
\begin{split}
  \mathcal{L}(\theta)
  &= 
  \underbrace{
  \tfrac{1}{HW}\!\sum_k\!\sum_{x,y}
  \bigl(\mathsf{E}_k(x,y;\theta)-\mu_k\bigr)^2
  }_{\mathcal{L}_{\mathrm{var}}} \\[3pt]
  &\quad + 
  \lambda_{\mathrm{sm}}
  \underbrace{
  \sum_{i=1}^{M-2}
  \!\Big[
  (\tilde{a}_{i+1}-\tilde{a}_i)^2
  +(\tilde{b}_{i+1}-\tilde{b}_i)^2
  \Big]
  }_{\mathcal{L}_{\mathrm{sm}}},
  \label{eq:loss}
\end{split}
\end{equation}
where $\mathsf{E}_k(x,y)$ is the compensated event frame in bin $k$, $\mu_k = (HW)^{-1}\sum_{x,y}\mathsf{E}_k(x,y)$ is its spatial mean, and $\lambda_{\mathrm{sm}}$ controls the smoothness strength. Minimizing $\mathcal{L}(\theta)$ by gradient descent aligns temporal windows, suppresses acceleration-induced shear, and yields sharper spectral reconstructions than a single-plane warp.

\begin{figure}[t]
  \centering
  % Figure 3(a): Events with learned boundaries (X--T and Y--T)
  \begin{tikzpicture}
    \node[anchor=south west, inner sep=0] (img3a) at (0,0)
      % {\includegraphics[width=\linewidth]{../../publication_code/figures/figure03_a_events.pdf}};
      {\includegraphics[width=1\linewidth]{figures/multiwindow_events_blank.pdf}};
    \begin{scope}[x={(img3a.south east)}, y={(img3a.north west)}]
      \node[font=\bfseries, anchor=north west, fill=none, text=black, inner sep=2pt] at (-0.05, 1.050) {(a)};
    \end{scope}
  \end{tikzpicture}\\[-0pt]
  % Figure 3(b): Variance vs time (50 ms bins)
  \begin{tikzpicture}
    \node[anchor=south west, inner sep=0] (img3b) at (0,0)
      % {\includegraphics[width=\linewidth]{../../publication_code/figures/figure03_b_variance.pdf}};
      {\includegraphics[width=1\linewidth]{figures/multiwindow_variance_blank.pdf}};
    \begin{scope}[x={(img3b.south east)}, y={(img3b.north west)}]
      \node[font=\bfseries, anchor=north west, fill=none, text=black, inner sep=2pt] at (-0.05, 1.050) {(b)};
    \end{scope}
  \end{tikzpicture}\\[-0pt]
  % Figure 3(c): Single 50 ms bin comparison (original vs compensated)
  \begin{tikzpicture}
    \node[anchor=south west, inner sep=0] (img3c) at (0,0)
      % {\includegraphics[width=\linewidth]{../../publication_code/figures/figure03_c_bin50ms.pdf}};
      {\includegraphics[width=1\linewidth]{figures/multiwindow_bin50ms_sanqin.pdf}};
    \begin{scope}[x={(img3c.south east)}, y={(img3c.north west)}]
      \node[font=\bfseries, anchor=north west, fill=none, text=black, inner sep=2pt] at (-0.05, 1.05) {(c)};
    \end{scope}
  \end{tikzpicture}\\[-0pt]
\caption{Supplementary diagnostics for multi-window compensation. (a) Events (X--T and Y--T) with learned boundary surfaces overlaid. (b) Variance per 50~ms bin over time (original vs. compensated). (c) Example 50~ms bin comparing original and compensated frames.}
  \label{fig:warp}
\end{figure}

Panel~(a) illustrates how the learned boundary surfaces align the event distribution in both spatial–temporal projections, effectively capturing the nonuniform scanning motion. 
The fitted slopes span from  \SI{-1.74}{\micro\second\per\pixel} to \SI{5.50}{\micro\second\per\pixel} for \(X\)-axis, and from \SI{-79.82}{\micro\second\per\pixel} to \SI{-75.67}{\micro\second\per\pixel} for \(Y\)-axis, representing the lateral and vertical time-per-pixel coefficients of the multi-window model.
Panel~(b) shows that the temporal variance within each 50~ms bin decreases after compensation, indicating improved temporal coherence across the field of view. 
Panel~(c) compares representative cumulative event frames before and after correction, where the removal of diagonal blur patterns confirms that the time-warp model successfully restores the spatial integrity of each spectral slice.

\olsection{Time--wavelength auto-alignment}
A one-dimensional background trace $b(t)$ is computed from polarity-weighted, fast-compensated events using a fixed step $\Delta t$ (5~ms). After normalising by the start and end plateaus,
\begin{equation}
  \tilde b(t)=\frac{b(t)-\mu_{\mathrm{pre}}}{\mu_{\mathrm{post}}-\mu_{\mathrm{pre}}},\quad t\in[0,T],
\end{equation}
the active scan interval $[t_0,t_1]$ is identified from the rising and falling edge quantiles of the smoothed $\tilde b(t)$: $t_0$ is the first time at which $\tilde b(t)$ crosses a lower edge level, and $t_1$ is the last time at which it crosses the corresponding upper edge level. This edges-only procedure robustly locks the start and tail plateaus. For ground-truth spectra $\{g_i(\lambda)\}_{i=1}^M$ with active intervals $[\lambda_{0,i},\lambda_{1,i}]$, the time-to-wavelength mapping is given by the affine model
\begin{equation}
  \lambda(t)=\alpha t+\beta,\quad 
  \alpha=\frac{\bar\lambda_1-\bar\lambda_0}{t_1-t_0},\;
  \beta=\bar\lambda_0-\alpha t_0,
\end{equation}
where $\bar\lambda_0$ and $\bar\lambda_1$ are the mean boundary wavelengths across curves. The transformed series $\tilde b_\lambda(\lambda)=\tilde b\big((\lambda-\beta)/\alpha\big)$ is then amplitude-aligned to the mean ground-truth $\bar g(\lambda)=M^{-1}\sum_i \tilde g_i(\lambda)$ by a least-squares affine fit,
\begin{equation}
  (a^*,c^*)=\arg\min_{a,c}\sum_j[\bar g(\lambda_j)-a\,\tilde b_\lambda(\lambda_j)-c]^2,
\end{equation}
with closed-form solution $[a^*,c^*]^\top=(X^\top X)^{-1}X^\top\mathbf{y}$, $X=[\tilde b_\lambda\;\mathbf{1}]$, $\mathbf{y}=\bar g$.


\bibliography{ref}

\end{document}
