\documentclass[9pt,twocolumn]{opticajnl}
\journal{opticajournal}
\setboolean{shortarticle}{true}
\usepackage{siunitx}
\sisetup{range-phrase = --, range-units = single}
\usepackage{siunitx}
\sisetup{per-mode = symbol}
\DeclareSIUnit{\pixel}{px}
\DeclareSIUnit{\USD}{USD}
\usepackage{ifthen}
\usepackage{placeins}
\usepackage{xcolor}
\usepackage{tikz}
\usetikzlibrary{arrows.meta,positioning}

% Keep typography and packages consistent with the main Optica letter
\IfFileExists{siunitx.sty}{%
  \usepackage{siunitx}
  \sisetup{range-phrase = --, range-units = single}
  \sisetup{per-mode = symbol}
}{%
  \newcommand{\sisetup}[1]{}
  \newcommand{\SI}[2]{\ensuremath{##1\,##2}}
  \newcommand{\SIrange}[3]{\ensuremath{##1\text{--}##2\,##3}}
  \newcommand{\DeclareSIUnit}[2]{}
}
\providecommand{\nano}{\ensuremath{\mathrm{n}}}
\providecommand{\micro}{\ensuremath{\mu}}
\providecommand{\milli}{\ensuremath{\mathrm{m}}}
\providecommand{\second}{\ensuremath{\mathrm{s}}}
\providecommand{\meter}{\ensuremath{\mathrm{m}}}
\providecommand{\per}{\ensuremath{\,/\,}}
\providecommand{\pixel}{\ensuremath{\mathrm{px}}}
\providecommand{\USD}{\ensuremath{\mathrm{USD}}}
\DeclareSIUnit{\pixel}{px}
\DeclareSIUnit{\USD}{USD}
\usepackage{xcolor}

\title{Supplementary Material: Self-Calibrated Neuromorphic Hyperspectral Sensing}
\author[]{Rongzhou Chen}
\author[]{Chutian Wang}
\author[]{Yuqing Cao}
\author[*]{Edmund Y. Lam}
\affil[]{Department of Electrical and Electronic Engineering, The University of Hong Kong, Pokfulam, Hong Kong SAR, China}
\affil[*]{elam@eee.hku.hk}
\dates{\today}

\begin{abstract}
This document provides supplementary methods, implementation notes, and additional figures supporting the main manuscript titled ``Self-Calibrated Neuromorphic Hyperspectral Sensing.'' 
\end{abstract}

\begin{document}

\maketitle

% Simple section helper matching main style
\newcommand{\olsection}[1]{\par\noindent\textbf{#1.} }

\olsection{Event-trigger probability model}
Event emission occurs when the logarithmic brightness increment surpasses a fixed magnitude $\theta$. Under realistic sensing conditions the observed gradient is corrupted by noise, which we model as
\begin{equation}
  \nabla I = \nabla I_{\text{true}} + \eta,\qquad \eta \sim \mathcal{N}(0,\sigma^2).
  \label{eq:gradient_noise}
\end{equation}

The probability of a positive event equals the likelihood that the noise-perturbed gradient exceeds the threshold,
\begin{equation}
  P(\text{event}) = P(\eta > \theta - \nabla I_{\text{true}}) = \tfrac12\!\left[1 - \operatorname{erf}\!\left(\frac{\theta - \nabla I_{\text{true}}}{\sqrt{2}\sigma}\right)\right],
  \label{eq:trigger_probability}
\end{equation}
with an analogous form for negative events around $-\theta$. Approximating the error function by a logistic curve yields the computationally convenient sigmoid expression
\begin{equation}
  P(\text{event}) \approx \sigma\!\left(\frac{\nabla I_{\text{true}} - \theta}{\sigma'}\right),\qquad \sigma' \approx \frac{2\sqrt{2}}{\sqrt{\pi}}\sigma,
  \label{eq:sigmoid_probability}
\end{equation}
where $\sigma(\cdot)$ denotes the logistic sigmoid. Within a temporal bin of duration $\Delta t$ the expected event count relates directly to this probability,
\begin{equation}
  \langle N_{\text{evt}}\rangle = \lambda\,P(\text{event})\,\Delta t,
  \label{eq:expected_events}
\end{equation}
with $\lambda$ the effective sampling rate per pixel. Averaging events over repeated scans gives the empirical mean $\bar{N}_{\text{evt}}$, which can be inverted via the logit function $\sigma^{-1}$ to estimate the true gradient:
\begin{equation}
  \nabla I_{\text{est}} = \theta + \sigma'\,\sigma^{-1}\!\left(\frac{\bar{N}_{\text{evt}}}{\lambda\,\Delta t}\right).
  \label{eq:gradient_estimate}
\end{equation}

This probabilistic link between event statistics and intensity gradients underpins the reconstruction strategy in the main letter. Equation~(\ref{eq:gradient_estimate}) justifies using temporally binned events as sufficient statistics for spectral estimation: the scan-induced evolution of $\nabla I_{\text{true}}$ encodes spectral slopes, while averaging suppresses sensor noise and preserves the dynamic-range benefits of event vision. Subsequent steps segment scans and correct residual temporal shear before building the spectral cube.

\olsection{Self-synchronized scan segmentation}
We first convert the event timestamps into an activity signal $a[n]$ (the number of events per 1~ms bin). The peak of auto-correlation
\begin{equation}
  R[k] = \sum_{n} a[n]\,a[n+k]
\end{equation}
reveals the scanning period, while the auto-convolution of $a[n]$ the correlation of $a[n]$ and its time-reversed sequence $a_{\mathrm{rev}}[n] = a[N{-}1{-}n]$,
\begin{equation}
  R_{\mathrm{rev}}[k] = \sum_{n} a[n]\,a[N{-}1{-}n],
  \label{eq:rev_corr}
\end{equation}
exposes the pre- and post-scan regions through the mirror-symmetric peaks around the turnaround time. An iterative peak search refines the start and end indices of each half-scan, producing scanning segments without reference measurements. Figure~\ref{fig:segmentation} illustrates the auto-correlation and auto-convolution used to estimate the scan period, as well as the resulting segmentation of forward and backward passes over the full recording. 

This data-driven segmentation adapts to slow drifts or perturbations, ensuring that subsequent reconstructions operate on consistent temporal support. 

\olsection{Multi-window scanning compensation}
Even with accurate segmentation, \eqref{eq:lambda_mapping} introduces a spatially varying delay through the $x/z_2$ term; motor acceleration near turnarounds further shears the time axis. Our multi-window compensation routine addresses this by fitting boundary surfaces
\begin{equation}
  T_i(x,y) = a_i x + b_i y + c_i,
\end{equation}
which partition each scan into $M$ temporal windows. Soft memberships,
\begin{equation}
  w_i(x,y,t) = \frac{\sigma\!\left(\frac{t - T_i}{\tau}\right)\sigma\!\left(\frac{T_{i+1}-t}{\tau}\right)}{\sum_j \sigma\!\left(\frac{t - T_j}{\tau}\right)\sigma\!\left(\frac{T_{j+1}-t}{\tau}\right)},
\end{equation}
blend adjacent windows (temperature $\tau \approx 1$~ms). The compensated time is
\begin{equation}
  t' = t - \sum_{i=1}^{M-1} w_i(x,y,t)\bigl(\tilde{a}_i x + \tilde{b}_i y \bigr),
  \label{eq:timewarp}
\end{equation}
where $\{\tilde{a}_i,\tilde{b}_i\}$ are trainable slopes optimized by minimizing the spatial variance of binned event intensities while enforcing smoothness across neighbouring windows. The objective is 
\begin{equation}
\begin{split}
  \mathcal{L}(\theta)
  &= 
  \underbrace{
  \tfrac{1}{HW}\!\sum_k\!\sum_{x,y}
  \bigl(\mathsf{E}_k(x,y;\theta)-\mu_k\bigr)^2
  }_{\mathcal{L}_{\mathrm{var}}} \\[3pt]
  &\quad + 
  \lambda_{\mathrm{sm}}
  \underbrace{
  \sum_{i=1}^{M-2}
  \!\Big[
  (\tilde{a}_{i+1}-\tilde{a}_i)^2
  +(\tilde{b}_{i+1}-\tilde{b}_i)^2
  \Big]
  }_{\mathcal{L}_{\mathrm{sm}}},
  \label{eq:loss}
\end{split}
\end{equation}
where $\mathsf{E}_k(x,y)$ is the compensated event frame in bin $k$, $\mu_k = (HW)^{-1}\sum_{x,y}\mathsf{E}_k(x,y)$ is its spatial mean, and $\lambda_{\mathrm{sm}}$ controls the smoothness strength. Minimizing $\mathcal{L}(\theta)$ by gradient descent aligns temporal windows, suppresses acceleration-induced shear, and yields sharper spectral reconstructions than a single-plane warp.

\begin{figure}[t]
  \centering
  % Figure 3(a): Events with learned boundaries (X--T and Y--T)
  \begin{tikzpicture}
    \node[anchor=south west, inner sep=0] (img3a) at (0,0)
      % {\includegraphics[width=\linewidth]{../../publication_code/figures/figure03_a_events.pdf}};
      {\includegraphics[width=1\linewidth]{figures/multiwindow_events_blank.pdf}};
    \begin{scope}[x={(img3a.south east)}, y={(img3a.north west)}]
      \node[font=\bfseries, anchor=north west, fill=none, text=black, inner sep=2pt] at (-0.05, 1.050) {(a)};
    \end{scope}
  \end{tikzpicture}\\[-0pt]
  % Figure 3(b): Variance vs time (50 ms bins)
  \begin{tikzpicture}
    \node[anchor=south west, inner sep=0] (img3b) at (0,0)
      % {\includegraphics[width=\linewidth]{../../publication_code/figures/figure03_b_variance.pdf}};
      {\includegraphics[width=1\linewidth]{figures/multiwindow_variance_blank.pdf}};
    \begin{scope}[x={(img3b.south east)}, y={(img3b.north west)}]
      \node[font=\bfseries, anchor=north west, fill=none, text=black, inner sep=2pt] at (-0.05, 1.050) {(b)};
    \end{scope}
  \end{tikzpicture}\\[-0pt]
  % Figure 3(c): Single 50 ms bin comparison (original vs compensated)
  \begin{tikzpicture}
    \node[anchor=south west, inner sep=0] (img3c) at (0,0)
      % {\includegraphics[width=\linewidth]{../../publication_code/figures/figure03_c_bin50ms.pdf}};
      {\includegraphics[width=1\linewidth]{figures/multiwindow_bin50ms_sanqin.pdf}};
    \begin{scope}[x={(img3c.south east)}, y={(img3c.north west)}]
      \node[font=\bfseries, anchor=north west, fill=none, text=black, inner sep=2pt] at (-0.05, 1.05) {(c)};
    \end{scope}
  \end{tikzpicture}\\[-0pt]
  \caption{Multi-window compensation diagnostics. (a) Events (X--T and Y--T) with learned boundary surfaces overlaid. (b) Variance per 50~ms bin over time (original vs. compensate). (c) Example 50~ms bin comparing original and compensate frames.}
  \label{fig:warp}
\end{figure}

Panel~(a) illustrates how the learned boundary surfaces align the event distribution in both spatial–temporal projections, effectively capturing the nonuniform scanning motion. 
The fitted slopes span from  \SI{-1.74}{\micro\second\per\pixel} to \SI{5.50}{\micro\second\per\pixel} for \(X\)-axis, and from \SI{-79.82}{\micro\second\per\pixel} to \SI{-75.67}{\micro\second\per\pixel} for \(Y\)-axis, representing the lateral and vertical time-per-pixel coefficients of the multi-window model.
Panel~(b) shows that the temporal variance within each 50~ms bin decreases after compensation, indicating improved temporal coherence across the field of view. 
Panel~(c) compares representative cumulative event frames before and after correction, where the removal of diagonal blur patterns confirms that the time-warp model successfully restores the spatial integrity of each spectral slice.

\olsection{Time--wavelength auto-alignment}
A one-dimensional background trace $b(t)$ is computed from polarity-weighted, fast-compensated events using a fixed step $\Delta t$ (5~ms). After normalising by the start and end plateaus,
\begin{equation}
  \tilde b(t)=\frac{b(t)-\mu_{\mathrm{pre}}}{\mu_{\mathrm{post}}-\mu_{\mathrm{pre}}},\quad t\in[0,T],
\end{equation}
the active scan interval $[t_0,t_1]$ is identified from the rising and falling edge quantiles of the smoothed $\tilde b(t)$: $t_0$ is the first time at which $\tilde b(t)$ crosses a lower edge level, and $t_1$ is the last time at which it crosses the corresponding upper edge level. This edges-only procedure robustly locks the start and tail plateaus. For ground-truth spectra $\{g_i(\lambda)\}_{i=1}^M$ with active intervals $[\lambda_{0,i},\lambda_{1,i}]$, the time-to-wavelength mapping is given by the affine model
\begin{equation}
  \lambda(t)=\alpha t+\beta,\quad 
  \alpha=\frac{\bar\lambda_1-\bar\lambda_0}{t_1-t_0},\;
  \beta=\bar\lambda_0-\alpha t_0,
\end{equation}
where $\bar\lambda_0$ and $\bar\lambda_1$ are the mean boundary wavelengths across curves. The transformed series $\tilde b_\lambda(\lambda)=\tilde b\big((\lambda-\beta)/\alpha\big)$ is then amplitude-aligned to the mean ground-truth $\bar g(\lambda)=M^{-1}\sum_i \tilde g_i(\lambda)$ by a least-squares affine fit,
\begin{equation}
  (a^*,c^*)=\arg\min_{a,c}\sum_j[\bar g(\lambda_j)-a\,\tilde b_\lambda(\lambda_j)-c]^2,
\end{equation}
with closed-form solution $[a^*,c^*]^\top=(X^\top X)^{-1}X^\top\mathbf{y}$, $X=[\tilde b_\lambda\;\mathbf{1}]$, $\mathbf{y}=\bar g$.


\bibliography{ref}

\end{document}
