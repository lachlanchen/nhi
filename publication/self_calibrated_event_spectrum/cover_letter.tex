\documentclass[11pt]{letter}
\usepackage[margin=1in]{geometry}
\usepackage{setspace}
\setstretch{1.05}

\begin{document}

\begin{letter}{Editors\\Optica}

\centerline{\large\textbf{Cover Letter for Manuscript:\\``Self-Calibrated Neuromorphic Hyperspectral Imaging''}}
\vspace{1em}

\opening{Dear Editor,}


We are pleased to submit our manuscript entitled \textbf{``Self-Calibrated Neuromorphic Hyperspectral Imaging''} for consideration in \textit{Optica}. The work presents a neuromorphic hyperspectral imaging system that combines a minimalist optical module with a fully data-driven event-based reconstruction pipeline. The scanner operates without encoder or trigger synchronization by deriving scan timing and segmentation directly from the event stream, while a dedicated correction algorithm accommodates the non-uniform motion of a low-cost motor. Together, these elements form a compact system that performs rapid hyperspectral scans while remaining mechanically simple, inexpensive, and easy to integrate with standard microscopy.

The manuscript details two core technical components that enable this self-calibrated operation. First, we establish a self-synchronization strategy based on auto-correlation of the event activity and its correlation with a time-reversed trace, allowing precise events segmentation without auxiliary electronics. Second, we introduce a multi-window time-warp compensation that corrects shear from non-uniform motor motion by optimizing window-wise temporal slopes to improve spatial coherence in binned events. These computational steps jointly stabilize the wavelength mapping and yield spectrally meaningful reconstructions using low-cost, non-uniform actuation.

Together, these contributions demonstrate a practical pathway toward accessible hyperspectral imaging: scans complete in under a second, data volumes are reduced relative to frame-based systems, and event-driven sensing maintains robustness under dim illumination where conventional cameras require long exposures. We believe this combination of self-calibration, computational correction, and hardware simplicity will interest the \textit{Optica} community and broaden the range of deployable, low-cost spectral imaging platforms. The manuscript is original, not under review elsewhere, and approved by all authors.


\closing{Sincerely,\\[0.5em]Edmund Y. Lam\\(on behalf of all authors)\\Department of Electrical and Electronic Engineering\\The University of Hong Kong\\\texttt{elam@eee.hku.hk}}

\end{letter}

\end{document}
