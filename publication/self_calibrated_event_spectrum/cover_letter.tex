\documentclass[11pt]{letter}
\usepackage[margin=1in]{geometry}
\usepackage{setspace}
\setstretch{1.05}

\begin{document}

\begin{letter}{Editors\\Optica}
\opening{Dear Editor,}

We are pleased to submit our manuscript \textbf{``Self-Calibrated Neuromorphic Hyperspectral Imaging''} for consideration in \textit{Optica}. The work introduces a novel event-based hyperspectral imaging system that achieves high spectral fidelity with drastically simplified hardware and algorithmic self-calibration. We believe it will be of strong interest to the optics and imaging community, especially given its emphasis on hardware simplicity, algorithmic innovation, low-light performance, and cost reduction.

\textbf{Hardware simplicity and cost-effectiveness.} The scanner uses only a low-cost LED, diffraction grating, and uncalibrated miniature motor (about USD~35, excluding the event sensor). No encoders or trigger wiring are required. The modular design integrates with microscopes via a simple 4$f$ relay and beam splitter, representing a dramatic cost reduction compared with typical USD~10k--15k hyperspectral cameras.

\textbf{Algorithmic novelty: self-synchronization and compensation.} We segment scanning passes using only the event data, computing the auto-correlation of millisecond-scale event activity and its correlation with the reversed trace to detect scan periods and separate forward/backward sweeps with no external timing reference. A multi-window time-warp compensation divides the scan into piecewise-linear temporal windows and optimizes their alignment, minimizing spatial variance in binned events to correct temporal shear from non-uniform motor motion. This maintains accurate wavelength alignment even with inexpensive, non-uniform actuation.

\textbf{High speed and low-light performance.} A full hyperspectral scan (400--700~nm) completes in 0.6--0.8~s, versus several minutes ($\approx$300~s) for a conventional frame-based hyperspectral camera. The event sensor’s high dynamic range and microsecond responsiveness enable reliable reconstruction under dim illumination where traditional cameras require long exposures or fail.

\textbf{Validated spectral fidelity.} Reconstructed spectra agree closely with a reference hyperspectral camera (e.g., plant stem cross-section absorption bands), even when the reference requires 300~s exposures and suffers low-light noise. While current results emphasize relative spectral intensity rather than absolute radiometry, the comparative agreement demonstrates the effectiveness of the self-calibration and compensation pipeline.

These contributions substantially lower the cost and complexity barriers to hyperspectral imaging and open pathways for portable, real-time instruments in field or resource-limited settings. The self-synchronization via event analysis and multi-window correction should interest readers in computational imaging and sensor fusion beyond the hyperspectral domain.

The manuscript is original, not under consideration elsewhere, and follows the Optica short-article format. All authors have approved the submission. Thank you for your consideration; we look forward to the reviewers’ feedback.

\closing{Sincerely,\\[0.5em]Edmund Y. Lam\\(on behalf of all authors)\\Department of Electrical and Electronic Engineering,\\The University of Hong Kong\\\texttt{elam@eee.hku.hk}}
\end{letter}

\end{document}
